\documentclass[titlepage]{article}
\usepackage{amsmath}
\usepackage{framed}
\usepackage[margin = 1in]{geometry}
\title{A Theoretical Approach to Physics and Competitions}
\author{Kevin Yang}
\date{}
\begin{document}
\maketitle

\section{Understanding How This Will Work}
\subsection{Structure}
Welcome! I hope that this will somehow help you. Every explanation of the topics will be fully theoretical and math based. 
\\
The explanations will be divided into chapters of different topics of physics including but not limited to mechanics, electricity, magnetism and special relativity. 
\\
Each topic may be further subdivided into pdfs inorder to ease understanding and reading. 
\\
There will often be problems and solutions at the end of each pdf as review and more problems within the pdfs as practice.
\\
Remember that physics builds on itself so its best to read the pdfs in order.

\subsection{Inside the Explanations}
This will explain what different thing within the explanation means.
\\
There will be a list of numbered equations in the introduction section. Some parts of the explanation will reference a number of a equation number.
\\
Equations will be boxed out and centered.
\begin{center}
	\begin{framed}
		Sample Equation
	\end{framed}
\end{center}

Important tips will have a circle and dot in front of it.

\begin{center}
	$\bigodot$ You should try to remember these tips.
\end{center}
\end{document}\documentclass[titlepage]{article}
\usepackage{amsmath}
\usepackage{framed}
\usepackage[margin = 1in]{geometry}
\title{A Theoretical Approach to Physics and Competitions}
\author{Kevin Yang}
\date{}
\begin{document}
\maketitle

\section{Understanding How This Will Work}
\subsection{Structure}
Welcome, I hope this entire thing will some how help you.

Every explanation of the topics will be fully theoretical and math based.

The explanations will be divided into pdfs of different topics of physics including but not limited to mechanics, electricity, magnetism and special relativity. 

Each topic may be further subdivided inorder to ease understanding and reading. 

There will often be problems and solutions at the end of each pdf as review and more problems within the pdfs as practice.

Remember that physics builds on itself so its best to read the pdfs in order.

\subsection{Inside the Explanations}
Equations will be boxed out and centered.
\begin{center}
	\begin{framed}
		Sample Equation
	\end{framed}
\end{center}

Important tips will have a circle and dot in front of it.

\begin{center}
	$\bigodot$ You should try to remember these tips.
\end{center}

Important vocabulary will be \textbf{bolded}.
\end{document}